\chapter{Introduction}
\label{chapter:introduction}


% BACKGROUND
Multi-axial force sensors, also referred to as multi-axis force sensors, are devices that can measure forces and torques in multiple directions or axes simultaneously (from one to six dimensions). 
These sensors find applications in various fields such as robotics, biomechanics, and industrial automation \cite{multi_axis_force_sensors_review}. 

% There are different types of multi-axial force sensors available, each utilizing different sensing technologies. 
% The strain-gauge sensor is commonly used in this field, but optical sensors have shown higher accuracy and linearity \cite{perfect_sensor}.

% Novelty/Research Gap/Unknown
The classical multi-axis force sensor is a system of uniaxial pressure sensors of different dimensions.
While the strain-gauge sensor remains the most common solution as uniaxial pressure measurement cell in the field of multi-axial force sensors, 
optical sensors shows noticeable accuracy with higher linearity \cite{perfect_sensor}. 
The structures of the optical multi-force sensors has been solid \cite{perfect_sensor, 1990_optic}, while for strain-gauge based sensors fully mechanically decoupled solutions exists \cite{decoupling_sliding_structure, modal_sensor}. 

% Questions/Problems/Purpose of study
The objective of this project is to investigate the relationship between the linearity of optical sensors 
and the construction of the barrier and multi-axial sensor configuration.
% Experimental/Design Approach
The project is divided in two parts: development of a single degree of freedom pressure sensor and a multi-axis force sensor based on a designed pressure cell.
Both parts aim to develop a mathematical model for the sensor, its prototype and conduct testing to evaluate accuracy, hysteresis and axis crosstalk.
% I decided to held experiments under normal conditions and calibrate the sensor under statical load.

% A bridge to LR
The multiaxis force sensor development is ...


The most common solutions for multi-axial force sensors constructions and pressure cells types are described in the  


% By doing so, we hope to contribute to the existing body of knowledge in the field of force sensor technology.


% Не актуально из-за высокой стоимости калибровки датчика. добавить про актуальность, цену датчика и шум, сказать что на оптопарах есть предположение 
% что можно сделать дешевле. В конце посчитать сколько стоит производство такого датчика (партии).

% прописать задачи^ мат модель, зависимость, разработать прототип одноосевого, проверить эксперементально модель, разработать шестиосевой. 
% Прототип на фотополимерном принтере.

% квадратный барьер, разработка экспериментов (калибровка в нормальных условиях, измерение точности и гистерезиса).
% после цели, задачи, плавный переход в литревью.
% на магистерскую оптимизация формы барьера с учетом элестостатических показателей статики формы крышки.

\section{Overview of thesis contents}

\nameref{chapter:literature_review} chapter will outline the general information about multi-axial force sensors, force measurement cells technologies and state pros and cons of multi-axial force sensors existing solutions. 
Also the chapter will review methods of statical calibration and experimental setups used for testing the sensor' kind.

\nameref{chapter:optical_modeling} chapter will provide the mathematical model of the optical force measuring cell.

Construction modeling will present the target multi-axis sensor structure.

In \nameref{chapter:implementation} chapter I describe the experimental setup and the results evaluation techniques, provide comparison with my mathematical model.

% \nameref{chapter:results_and_discussion} chapter will outline the outcome of the sensor and discuss the future possibilities of the project.
% Java Overview chapter will describe the source language structure and most notable features that matter for the project.

% EO Overview chapter will outline the target language structure and most notable features that matter for the project.

% Implementation chapter will provide projecting methodology in depth, as well as describe the project implementation details.

% Results and Discussion chapter will present the outcome of the implemented project and discuss the future possibilities and relevance of the project.
